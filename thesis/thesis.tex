\documentclass[a4paper,11pt]{report}
\usepackage{preamble}

\begin{document}

\pagenumbering{roman}

% ======== TITLE ========
% \maketitle
\begin{titlepage}
    \begin{center}
        \LARGE

        VIETNAMESE - GERMAN UNIVERSITY\\
        DEPARTMENT OF COMPUTER SCIENCE

        \vspace*{0.5cm}

        Frankfurt University Of Applied Sciences\\
        Faculty 2: Computer Science and Engineering

        \vspace*{2cm}

        \Large

        \textbf{Thesis topic:}\\
        Assessing Covid-19 Situation in Vietnam\\
        Using a Data-Driven Epidemiological Compartmental Model

        \vspace*{2cm}

        \large

        Full name: Vo Le Tung\\
        Matriculation number: 1276244

        \vspace*{0.5cm}

        Fist supervisor: Assoc. Prof. Huynh Trung Hieu\\
        Second supervisor: Assoc. Prof. Nguyen Tuan Duc

        \vspace*{2cm}

        \LARGE
        \textbf{BACHELOR THESIS}

        \vspace*{2cm}

        \large
        Submitted in partial fulfillment of the requirements for\\
        the Bachelor of Science degree in Computer Science\\
        at the Vietnamese - German University

        \vfill

        6th January 2022, Binh Duong, Vietnam
    \end{center}
\end{titlepage}

% ======== ABSTRACT ========
\begin{abstract}
The Covid-19 disease that emerged in 2019 had evolved at a fast pace and affected millions of lives both medically and economically.
In the efforts to learn more about the disease and help to inform governments on policies making process, researchers had proposed various modeling techniques that can help with forecasting the evolution of the disease and estimating the different effects of government interventions in slowing down the spread of the virus.
Nonetheless, not many studies had been conducted with data for Vietnam.
We identified a gap in the knowledge about the dynamics of Covid-19 in Vietnam as well as the lack of study on the effects of government interventions on containing disease outbreaks.
In addition, the recent failure in suppressing an outbreak in Vietnam after a long period of successfully keeping a low number of infections had proven that there was more to learn about Covid-19, and a model that was specifically tailored for the data availability in Vietnam could be an important tool for such task.

We implemented an infectious disease model for Covid-19 that integrated machine learning techniques and compartmental modeling for assessing the situations of the disease and for forecasting future progression.
Additionally, we tried to improve the model forecast performance by incorporating mobility data and social network connections data as covariates in the model since the method has been proven to be effective by state-of-the-art model for Covid-19.
We then analyzed and compared the performance of the model when different covariates were used to study whether they could improve the model forecast performance.
By design, the model was explainable as it demonstrated how the different compartments developed, and domain-specific insights could be gained from the model because it was based on well-proven epidemiological knowledge.
The explainability of the model is one desirable characteristic that ensures the model's credibility to epidemiologists and instills confidence in the end-users of the model.
The model can be applied for different geographic resolutions, and we demonstrated it for Vietnam, the United States, provinces in Vietnam, and counties in the United States.
We showed that the model was capable of learning valuable insights about Covid-19 and was able to make forecasts with high accuracy in many cases.
\end{abstract}

% ======== DECLARATION ========
\chapter*{Declaration}
I, Vo Le Tung, hereby declare that this bachelor thesis is a product of my own work, unless otherwise stated.
I further declare that the thesis has not been previously or concurrently submitted for evaluation at any other institutions.

\vspace{1cm}
\noindent 6th January 2022\\
\textbf{Signature}
\vspace{3cm}\\
\underline{\hspace{5cm}}\\

% ======== ACKNOWLEDGEMENT ========
\chapter*{Acknowledgement}
I would like to show my gratitude to Prof. Huynh Trung Hieu for having guided me through the process of completing this thesis.
His inputs have been invaluable and have helped me learn a lot more about the topic of the thesis.
Additionally, I would like to thank my family, my friends, and all the faculty members for their support during my time at the Vietnamese - German University.
% ======== TABLE OF CONTENTS ========
\tableofcontents

% ======== LIST OF FIGURES ========
\listoffigures

% ======== LIST OF TABLES ========
\listoftables

% ======== LIST OF ALGORITHMS ========
\listofalgorithms

\newpage
\pagenumbering{arabic}

% ======== INTRODUCTION ========
\section{Results}

\subsection{Model's outputs for Vietnam and the United States}

\begin{frame}{Forecasts: Country-level data}
    \begin{figure}[!htb]
        \centering
        \includegraphics[scale=0.2]{pred_country_level.pdf}
        \caption{Forecasts made by different versions of the model}
    \end{figure}
\end{frame}

\begin{frame}{$R_0$ and fatality rate: Country-level data}
    \begin{figure}[!htb]
        \centering
        \subcaptionbox{Viet nam}{\includegraphics[width=0.4\linewidth]{Re_and_fatality_vietnam.pdf}}
        \subcaptionbox{United States}{\includegraphics[width=0.4\linewidth]{Re_and_fatality_unitedstates.pdf}}
        \caption{Disease metrics learned by different versions of the model}
    \end{figure}
\end{frame}

\subsection{Model's outputs for counties in the United States}

\begin{frame}{Forecasts: Counties in the United States}
    \begin{figure}[!htb]
        \centering
        \includegraphics[scale=0.2]{pred_us_counties1.pdf}
        \includegraphics[scale=0.2]{pred_us_counties2.pdf}
        \caption{Forecasts made by different versions of the model}
    \end{figure}
\end{frame}

\begin{frame}{$R_0$ and fatality rate: Counties in the United States}
    \begin{figure}[!htb]
        \centering
        \subcaptionbox{Harris, TX}{\includegraphics[width=0.4\linewidth]{Re_and_fatality_harris_tx.pdf}}
        \subcaptionbox{Los Angeles, CA}{\includegraphics[width=0.4\linewidth]{Re_and_fatality_losangeles_ca.pdf}}
        \subcaptionbox{Maricopa, AZ}{\includegraphics[width=0.4\linewidth]{Re_and_fatality_maricopa_az.pdf}}
        \subcaptionbox{Cook, IL}{\includegraphics[width=0.4\linewidth]{Re_and_fatality_cook_il.pdf}}
        \caption{Disease metrics learned by different versions of the model}
    \end{figure}
\end{frame}

\subsection{Model's outputs for provinces in Vietnam}

\begin{frame}{Forecasts: Provinces in Vietnam}
    \begin{figure}[!htb]
        \centering
        \includegraphics[scale=0.15]{pred_vn_provinces1.pdf}
        \includegraphics[scale=0.15]{pred_vn_provinces2.pdf}
        \caption{Forecasts made by different versions of the model}
    \end{figure}
\end{frame}

\begin{frame}{$R_0$ and fatality rate: Provinces in Vietnam}
    \begin{figure}[!htb]
        \centering
        \subcaptionbox{Dong Nai}{\includegraphics[width=0.4\linewidth]{Re_and_fatality_dongnai.pdf}}
        \subcaptionbox{Long An}{\includegraphics[width=0.4\linewidth]{Re_and_fatality_longan.pdf}}
        \subcaptionbox{Binh Duong}{\includegraphics[width=0.4\linewidth]{Re_and_fatality_binhduong.pdf}}
        \subcaptionbox{Ho Chi Minh city}{\includegraphics[width=0.4\linewidth]{Re_and_fatality_hcm.pdf}}
        \caption{Disease metrics learned by different versions of the model}
    \end{figure}
\end{frame}


% ======== LITERATURE REVIEW ========
\section{Results}

\subsection{Model's outputs for Vietnam and the United States}

\begin{frame}{Forecasts: Country-level data}
    \begin{figure}[!htb]
        \centering
        \includegraphics[scale=0.2]{pred_country_level.pdf}
        \caption{Forecasts made by different versions of the model}
    \end{figure}
\end{frame}

\begin{frame}{$R_0$ and fatality rate: Country-level data}
    \begin{figure}[!htb]
        \centering
        \subcaptionbox{Viet nam}{\includegraphics[width=0.4\linewidth]{Re_and_fatality_vietnam.pdf}}
        \subcaptionbox{United States}{\includegraphics[width=0.4\linewidth]{Re_and_fatality_unitedstates.pdf}}
        \caption{Disease metrics learned by different versions of the model}
    \end{figure}
\end{frame}

\subsection{Model's outputs for counties in the United States}

\begin{frame}{Forecasts: Counties in the United States}
    \begin{figure}[!htb]
        \centering
        \includegraphics[scale=0.2]{pred_us_counties1.pdf}
        \includegraphics[scale=0.2]{pred_us_counties2.pdf}
        \caption{Forecasts made by different versions of the model}
    \end{figure}
\end{frame}

\begin{frame}{$R_0$ and fatality rate: Counties in the United States}
    \begin{figure}[!htb]
        \centering
        \subcaptionbox{Harris, TX}{\includegraphics[width=0.4\linewidth]{Re_and_fatality_harris_tx.pdf}}
        \subcaptionbox{Los Angeles, CA}{\includegraphics[width=0.4\linewidth]{Re_and_fatality_losangeles_ca.pdf}}
        \subcaptionbox{Maricopa, AZ}{\includegraphics[width=0.4\linewidth]{Re_and_fatality_maricopa_az.pdf}}
        \subcaptionbox{Cook, IL}{\includegraphics[width=0.4\linewidth]{Re_and_fatality_cook_il.pdf}}
        \caption{Disease metrics learned by different versions of the model}
    \end{figure}
\end{frame}

\subsection{Model's outputs for provinces in Vietnam}

\begin{frame}{Forecasts: Provinces in Vietnam}
    \begin{figure}[!htb]
        \centering
        \includegraphics[scale=0.15]{pred_vn_provinces1.pdf}
        \includegraphics[scale=0.15]{pred_vn_provinces2.pdf}
        \caption{Forecasts made by different versions of the model}
    \end{figure}
\end{frame}

\begin{frame}{$R_0$ and fatality rate: Provinces in Vietnam}
    \begin{figure}[!htb]
        \centering
        \subcaptionbox{Dong Nai}{\includegraphics[width=0.4\linewidth]{Re_and_fatality_dongnai.pdf}}
        \subcaptionbox{Long An}{\includegraphics[width=0.4\linewidth]{Re_and_fatality_longan.pdf}}
        \subcaptionbox{Binh Duong}{\includegraphics[width=0.4\linewidth]{Re_and_fatality_binhduong.pdf}}
        \subcaptionbox{Ho Chi Minh city}{\includegraphics[width=0.4\linewidth]{Re_and_fatality_hcm.pdf}}
        \caption{Disease metrics learned by different versions of the model}
    \end{figure}
\end{frame}


% ======== RESULTS ========
\section{Results}

\subsection{Model's outputs for Vietnam and the United States}

\begin{frame}{Forecasts: Country-level data}
    \begin{figure}[!htb]
        \centering
        \includegraphics[scale=0.2]{pred_country_level.pdf}
        \caption{Forecasts made by different versions of the model}
    \end{figure}
\end{frame}

\begin{frame}{$R_0$ and fatality rate: Country-level data}
    \begin{figure}[!htb]
        \centering
        \subcaptionbox{Viet nam}{\includegraphics[width=0.4\linewidth]{Re_and_fatality_vietnam.pdf}}
        \subcaptionbox{United States}{\includegraphics[width=0.4\linewidth]{Re_and_fatality_unitedstates.pdf}}
        \caption{Disease metrics learned by different versions of the model}
    \end{figure}
\end{frame}

\subsection{Model's outputs for counties in the United States}

\begin{frame}{Forecasts: Counties in the United States}
    \begin{figure}[!htb]
        \centering
        \includegraphics[scale=0.2]{pred_us_counties1.pdf}
        \includegraphics[scale=0.2]{pred_us_counties2.pdf}
        \caption{Forecasts made by different versions of the model}
    \end{figure}
\end{frame}

\begin{frame}{$R_0$ and fatality rate: Counties in the United States}
    \begin{figure}[!htb]
        \centering
        \subcaptionbox{Harris, TX}{\includegraphics[width=0.4\linewidth]{Re_and_fatality_harris_tx.pdf}}
        \subcaptionbox{Los Angeles, CA}{\includegraphics[width=0.4\linewidth]{Re_and_fatality_losangeles_ca.pdf}}
        \subcaptionbox{Maricopa, AZ}{\includegraphics[width=0.4\linewidth]{Re_and_fatality_maricopa_az.pdf}}
        \subcaptionbox{Cook, IL}{\includegraphics[width=0.4\linewidth]{Re_and_fatality_cook_il.pdf}}
        \caption{Disease metrics learned by different versions of the model}
    \end{figure}
\end{frame}

\subsection{Model's outputs for provinces in Vietnam}

\begin{frame}{Forecasts: Provinces in Vietnam}
    \begin{figure}[!htb]
        \centering
        \includegraphics[scale=0.15]{pred_vn_provinces1.pdf}
        \includegraphics[scale=0.15]{pred_vn_provinces2.pdf}
        \caption{Forecasts made by different versions of the model}
    \end{figure}
\end{frame}

\begin{frame}{$R_0$ and fatality rate: Provinces in Vietnam}
    \begin{figure}[!htb]
        \centering
        \subcaptionbox{Dong Nai}{\includegraphics[width=0.4\linewidth]{Re_and_fatality_dongnai.pdf}}
        \subcaptionbox{Long An}{\includegraphics[width=0.4\linewidth]{Re_and_fatality_longan.pdf}}
        \subcaptionbox{Binh Duong}{\includegraphics[width=0.4\linewidth]{Re_and_fatality_binhduong.pdf}}
        \subcaptionbox{Ho Chi Minh city}{\includegraphics[width=0.4\linewidth]{Re_and_fatality_hcm.pdf}}
        \caption{Disease metrics learned by different versions of the model}
    \end{figure}
\end{frame}


% ======== RESULTS ========
\section{Results}

\subsection{Model's outputs for Vietnam and the United States}

\begin{frame}{Forecasts: Country-level data}
    \begin{figure}[!htb]
        \centering
        \includegraphics[scale=0.2]{pred_country_level.pdf}
        \caption{Forecasts made by different versions of the model}
    \end{figure}
\end{frame}

\begin{frame}{$R_0$ and fatality rate: Country-level data}
    \begin{figure}[!htb]
        \centering
        \subcaptionbox{Viet nam}{\includegraphics[width=0.4\linewidth]{Re_and_fatality_vietnam.pdf}}
        \subcaptionbox{United States}{\includegraphics[width=0.4\linewidth]{Re_and_fatality_unitedstates.pdf}}
        \caption{Disease metrics learned by different versions of the model}
    \end{figure}
\end{frame}

\subsection{Model's outputs for counties in the United States}

\begin{frame}{Forecasts: Counties in the United States}
    \begin{figure}[!htb]
        \centering
        \includegraphics[scale=0.2]{pred_us_counties1.pdf}
        \includegraphics[scale=0.2]{pred_us_counties2.pdf}
        \caption{Forecasts made by different versions of the model}
    \end{figure}
\end{frame}

\begin{frame}{$R_0$ and fatality rate: Counties in the United States}
    \begin{figure}[!htb]
        \centering
        \subcaptionbox{Harris, TX}{\includegraphics[width=0.4\linewidth]{Re_and_fatality_harris_tx.pdf}}
        \subcaptionbox{Los Angeles, CA}{\includegraphics[width=0.4\linewidth]{Re_and_fatality_losangeles_ca.pdf}}
        \subcaptionbox{Maricopa, AZ}{\includegraphics[width=0.4\linewidth]{Re_and_fatality_maricopa_az.pdf}}
        \subcaptionbox{Cook, IL}{\includegraphics[width=0.4\linewidth]{Re_and_fatality_cook_il.pdf}}
        \caption{Disease metrics learned by different versions of the model}
    \end{figure}
\end{frame}

\subsection{Model's outputs for provinces in Vietnam}

\begin{frame}{Forecasts: Provinces in Vietnam}
    \begin{figure}[!htb]
        \centering
        \includegraphics[scale=0.15]{pred_vn_provinces1.pdf}
        \includegraphics[scale=0.15]{pred_vn_provinces2.pdf}
        \caption{Forecasts made by different versions of the model}
    \end{figure}
\end{frame}

\begin{frame}{$R_0$ and fatality rate: Provinces in Vietnam}
    \begin{figure}[!htb]
        \centering
        \subcaptionbox{Dong Nai}{\includegraphics[width=0.4\linewidth]{Re_and_fatality_dongnai.pdf}}
        \subcaptionbox{Long An}{\includegraphics[width=0.4\linewidth]{Re_and_fatality_longan.pdf}}
        \subcaptionbox{Binh Duong}{\includegraphics[width=0.4\linewidth]{Re_and_fatality_binhduong.pdf}}
        \subcaptionbox{Ho Chi Minh city}{\includegraphics[width=0.4\linewidth]{Re_and_fatality_hcm.pdf}}
        \caption{Disease metrics learned by different versions of the model}
    \end{figure}
\end{frame}


% ======== DISCUSSION ========
\section{Results}

\subsection{Model's outputs for Vietnam and the United States}

\begin{frame}{Forecasts: Country-level data}
    \begin{figure}[!htb]
        \centering
        \includegraphics[scale=0.2]{pred_country_level.pdf}
        \caption{Forecasts made by different versions of the model}
    \end{figure}
\end{frame}

\begin{frame}{$R_0$ and fatality rate: Country-level data}
    \begin{figure}[!htb]
        \centering
        \subcaptionbox{Viet nam}{\includegraphics[width=0.4\linewidth]{Re_and_fatality_vietnam.pdf}}
        \subcaptionbox{United States}{\includegraphics[width=0.4\linewidth]{Re_and_fatality_unitedstates.pdf}}
        \caption{Disease metrics learned by different versions of the model}
    \end{figure}
\end{frame}

\subsection{Model's outputs for counties in the United States}

\begin{frame}{Forecasts: Counties in the United States}
    \begin{figure}[!htb]
        \centering
        \includegraphics[scale=0.2]{pred_us_counties1.pdf}
        \includegraphics[scale=0.2]{pred_us_counties2.pdf}
        \caption{Forecasts made by different versions of the model}
    \end{figure}
\end{frame}

\begin{frame}{$R_0$ and fatality rate: Counties in the United States}
    \begin{figure}[!htb]
        \centering
        \subcaptionbox{Harris, TX}{\includegraphics[width=0.4\linewidth]{Re_and_fatality_harris_tx.pdf}}
        \subcaptionbox{Los Angeles, CA}{\includegraphics[width=0.4\linewidth]{Re_and_fatality_losangeles_ca.pdf}}
        \subcaptionbox{Maricopa, AZ}{\includegraphics[width=0.4\linewidth]{Re_and_fatality_maricopa_az.pdf}}
        \subcaptionbox{Cook, IL}{\includegraphics[width=0.4\linewidth]{Re_and_fatality_cook_il.pdf}}
        \caption{Disease metrics learned by different versions of the model}
    \end{figure}
\end{frame}

\subsection{Model's outputs for provinces in Vietnam}

\begin{frame}{Forecasts: Provinces in Vietnam}
    \begin{figure}[!htb]
        \centering
        \includegraphics[scale=0.15]{pred_vn_provinces1.pdf}
        \includegraphics[scale=0.15]{pred_vn_provinces2.pdf}
        \caption{Forecasts made by different versions of the model}
    \end{figure}
\end{frame}

\begin{frame}{$R_0$ and fatality rate: Provinces in Vietnam}
    \begin{figure}[!htb]
        \centering
        \subcaptionbox{Dong Nai}{\includegraphics[width=0.4\linewidth]{Re_and_fatality_dongnai.pdf}}
        \subcaptionbox{Long An}{\includegraphics[width=0.4\linewidth]{Re_and_fatality_longan.pdf}}
        \subcaptionbox{Binh Duong}{\includegraphics[width=0.4\linewidth]{Re_and_fatality_binhduong.pdf}}
        \subcaptionbox{Ho Chi Minh city}{\includegraphics[width=0.4\linewidth]{Re_and_fatality_hcm.pdf}}
        \caption{Disease metrics learned by different versions of the model}
    \end{figure}
\end{frame}


% ======== CONCLUSION ========
\section{Results}

\subsection{Model's outputs for Vietnam and the United States}

\begin{frame}{Forecasts: Country-level data}
    \begin{figure}[!htb]
        \centering
        \includegraphics[scale=0.2]{pred_country_level.pdf}
        \caption{Forecasts made by different versions of the model}
    \end{figure}
\end{frame}

\begin{frame}{$R_0$ and fatality rate: Country-level data}
    \begin{figure}[!htb]
        \centering
        \subcaptionbox{Viet nam}{\includegraphics[width=0.4\linewidth]{Re_and_fatality_vietnam.pdf}}
        \subcaptionbox{United States}{\includegraphics[width=0.4\linewidth]{Re_and_fatality_unitedstates.pdf}}
        \caption{Disease metrics learned by different versions of the model}
    \end{figure}
\end{frame}

\subsection{Model's outputs for counties in the United States}

\begin{frame}{Forecasts: Counties in the United States}
    \begin{figure}[!htb]
        \centering
        \includegraphics[scale=0.2]{pred_us_counties1.pdf}
        \includegraphics[scale=0.2]{pred_us_counties2.pdf}
        \caption{Forecasts made by different versions of the model}
    \end{figure}
\end{frame}

\begin{frame}{$R_0$ and fatality rate: Counties in the United States}
    \begin{figure}[!htb]
        \centering
        \subcaptionbox{Harris, TX}{\includegraphics[width=0.4\linewidth]{Re_and_fatality_harris_tx.pdf}}
        \subcaptionbox{Los Angeles, CA}{\includegraphics[width=0.4\linewidth]{Re_and_fatality_losangeles_ca.pdf}}
        \subcaptionbox{Maricopa, AZ}{\includegraphics[width=0.4\linewidth]{Re_and_fatality_maricopa_az.pdf}}
        \subcaptionbox{Cook, IL}{\includegraphics[width=0.4\linewidth]{Re_and_fatality_cook_il.pdf}}
        \caption{Disease metrics learned by different versions of the model}
    \end{figure}
\end{frame}

\subsection{Model's outputs for provinces in Vietnam}

\begin{frame}{Forecasts: Provinces in Vietnam}
    \begin{figure}[!htb]
        \centering
        \includegraphics[scale=0.15]{pred_vn_provinces1.pdf}
        \includegraphics[scale=0.15]{pred_vn_provinces2.pdf}
        \caption{Forecasts made by different versions of the model}
    \end{figure}
\end{frame}

\begin{frame}{$R_0$ and fatality rate: Provinces in Vietnam}
    \begin{figure}[!htb]
        \centering
        \subcaptionbox{Dong Nai}{\includegraphics[width=0.4\linewidth]{Re_and_fatality_dongnai.pdf}}
        \subcaptionbox{Long An}{\includegraphics[width=0.4\linewidth]{Re_and_fatality_longan.pdf}}
        \subcaptionbox{Binh Duong}{\includegraphics[width=0.4\linewidth]{Re_and_fatality_binhduong.pdf}}
        \subcaptionbox{Ho Chi Minh city}{\includegraphics[width=0.4\linewidth]{Re_and_fatality_hcm.pdf}}
        \caption{Disease metrics learned by different versions of the model}
    \end{figure}
\end{frame}


% ======== REFERENCES ========
\printbibliography[title=References]

% ======== REFERENCES ========
\section{Results}

\subsection{Model's outputs for Vietnam and the United States}

\begin{frame}{Forecasts: Country-level data}
    \begin{figure}[!htb]
        \centering
        \includegraphics[scale=0.2]{pred_country_level.pdf}
        \caption{Forecasts made by different versions of the model}
    \end{figure}
\end{frame}

\begin{frame}{$R_0$ and fatality rate: Country-level data}
    \begin{figure}[!htb]
        \centering
        \subcaptionbox{Viet nam}{\includegraphics[width=0.4\linewidth]{Re_and_fatality_vietnam.pdf}}
        \subcaptionbox{United States}{\includegraphics[width=0.4\linewidth]{Re_and_fatality_unitedstates.pdf}}
        \caption{Disease metrics learned by different versions of the model}
    \end{figure}
\end{frame}

\subsection{Model's outputs for counties in the United States}

\begin{frame}{Forecasts: Counties in the United States}
    \begin{figure}[!htb]
        \centering
        \includegraphics[scale=0.2]{pred_us_counties1.pdf}
        \includegraphics[scale=0.2]{pred_us_counties2.pdf}
        \caption{Forecasts made by different versions of the model}
    \end{figure}
\end{frame}

\begin{frame}{$R_0$ and fatality rate: Counties in the United States}
    \begin{figure}[!htb]
        \centering
        \subcaptionbox{Harris, TX}{\includegraphics[width=0.4\linewidth]{Re_and_fatality_harris_tx.pdf}}
        \subcaptionbox{Los Angeles, CA}{\includegraphics[width=0.4\linewidth]{Re_and_fatality_losangeles_ca.pdf}}
        \subcaptionbox{Maricopa, AZ}{\includegraphics[width=0.4\linewidth]{Re_and_fatality_maricopa_az.pdf}}
        \subcaptionbox{Cook, IL}{\includegraphics[width=0.4\linewidth]{Re_and_fatality_cook_il.pdf}}
        \caption{Disease metrics learned by different versions of the model}
    \end{figure}
\end{frame}

\subsection{Model's outputs for provinces in Vietnam}

\begin{frame}{Forecasts: Provinces in Vietnam}
    \begin{figure}[!htb]
        \centering
        \includegraphics[scale=0.15]{pred_vn_provinces1.pdf}
        \includegraphics[scale=0.15]{pred_vn_provinces2.pdf}
        \caption{Forecasts made by different versions of the model}
    \end{figure}
\end{frame}

\begin{frame}{$R_0$ and fatality rate: Provinces in Vietnam}
    \begin{figure}[!htb]
        \centering
        \subcaptionbox{Dong Nai}{\includegraphics[width=0.4\linewidth]{Re_and_fatality_dongnai.pdf}}
        \subcaptionbox{Long An}{\includegraphics[width=0.4\linewidth]{Re_and_fatality_longan.pdf}}
        \subcaptionbox{Binh Duong}{\includegraphics[width=0.4\linewidth]{Re_and_fatality_binhduong.pdf}}
        \subcaptionbox{Ho Chi Minh city}{\includegraphics[width=0.4\linewidth]{Re_and_fatality_hcm.pdf}}
        \caption{Disease metrics learned by different versions of the model}
    \end{figure}
\end{frame}


% ======== GLOSSARY ========
\printglossaries

\end{document}
