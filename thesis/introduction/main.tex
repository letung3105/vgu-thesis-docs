\chapter{Introduction}
\label{chap:introduction}

Since its emergence in 2019, the Covid-19 pandemic has evolved rapidly and is still affecting millions of lives around the globe.
According to data from the World Health Organization (WHO) \cite{WHOCoronavirusCOVID19}, Covid-19 has caused over 230 million infections and over 4.7 million death worldwide .
Despite the high infection rate across the world, Vietnam had been fortunate during most of the pandemic's period, and the country was recognized for its effective government's interventions in reducing the number of cases.
Until the first quarter of 2021, decisive actions and policies from the Vietnamese government have proven successful in containing the spread of Covid-19.
But the situation has quickly changed since the end of April 2021, when a new wave of infection hit the country.
The Delta variant of the SARS-CoV-2 virus, which is responsible for this infection wave, has a shorter incubation period and can spread quicker than the previous year \cite{mahaseDeltaVariantWhat2021}.
As a result, policies makers in Vietnam were caught off-guard, and the number of infections has been growing exponentially for the last few months (May 2021 - September 2021).
In response to the escalating number of daily reported cases, the Vietnamese government has enacted similar strategies as in 2020 and more stringent policies in regions with a high infection rate.
Unlike before, these aggressive responses to the virus spread where all citizens had to stay at home were not as effective in containing the spread of the virus.
At the time of this writing (October 2021), the number of daily infections is on the decline but remains high, while prolonged restrictions on movement have negatively impacted the livelihood of lots of people and the entire economy \cite{RapidAssessmentDesign}.

Like many other countries, one major factor that leads to the failure in swiftly containing the virus is the lack of knowledge in the dynamics of the pandemic under the effects of new variants and government interventions.
Researchers have proposed diverse methods for modeling the disease to understand these dynamics.
These methods use different techniques to investigate the strategies that can reduce the number of daily new infections.
\citeauthor{rahimiReviewCOVID19Forecasting2021} \cite{rahimiReviewCOVID19Forecasting2021} gave a systematic review of these methods.
Governments in different countries have employed these methods to aid in policy-making \cite{adamSpecialReportSimulations2020}.
One popular method is an ensemble of multiple models submitted by researchers \cite{rayEnsembleForecastsCoronavirus2020}, used by the \gls{US} Center for Disease Control and Prevention.

Although countries have been using these modeling and forecasting techniques to analyze the Covid-19 situation, many are not applied to Vietnam.
One of the reasons is because there was not a high number of recorded cases.
Thus the specific dynamics of Covid-19 in Vietnam have not been well studied.
Admit policies from other countries can be adopted; the effectiveness may vary due to socioeconomic, demographic, and cultural factors.
As the number of reported cases is dropping, the Vietnamese government is slowly removing its restrictions in many places to help improve the current economic recession.
Hence having a model for the distinct circumstances and data availability in Vietnam can be beneficial for assessing the pandemic's situation, especially when restrictions are lifted.
The model could be of help in the following aspects: (1) informing policies makers about the effects of their decision, (2) informing health care facilities about a possible surge in the number of cases to ensure the supply of personnel and equipment, (3) informing business owners about possible policies to plan their supply and demand needs.
Lastly, the recent surge in infections has generated data on the number of Covid-19 cases in Vietnam in a much higher quantity and quality that are publicly available and can be used for data-driven modeling.

Given that reasoning, this thesis focuses on implementing an interpretable disease model for Vietnam that can capture the current trends in the development of the pandemic at an early stage.
The implemented model will base on classical compartmental models where the mentioned issues are alleviated by using covariates to control the parameter(s).
Instead of using a predefined multivariate function to adjust the parameters in the system of \glspl{ODE}, we will be using an \gls{ANN} to encode the covariates.
The method takes advantage of the recent development in \gls{ANN} architectures for solving forward-inverse problems with \glspl{ODE} \cite{raissiPhysicsinformedNeuralNetworks2019, chenNeuralOrdinaryDifferential2019, rackauckasUniversalDifferentialEquations2020}.
Utilizing the capability of \glspl{ANN} to approximate any arbitrary function \cite{cybenkotApproximationSuperpositionsSigmoidal, hornikApproximationCapabilitiesMultilayer1991, hornikMultilayerFeedforwardNetworks1989}, data-driven approaches can discover the underlying mechanics of Covid-19 without needing to define the governing multivariate function, which requires expert epidemiological knowledge.

The rest of the thesis is structured as follow:
\begin{itemize}
    \item \autoref{chap:literature-review} gives an overview of the research backgrounds of the model.
    A list of related methods and techniques from other researchers is also presented.
    \item \autoref{chap:methodologies} discusses how the model was formulated, the used dataset, and the evaluation methods.
    \item \autoref{chap:results} presents our findings and evaluations for the performance of the implemented model.
    \item \autoref{chap:discussion} compares the results of the implemented model and its performance with related works produced by other researchers.This chapter includes a list of limitations of the model and further improvements that can be made to boost the effectiveness and performance of the model.
    \item \autoref{chap:conclusion} shows the overall achievement of this thesis.
\end{itemize}