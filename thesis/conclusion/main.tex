\chapter{Conclusion}
\label{chap:conclusion}

In this thesis, we implemented an infectious disease model that was enhanced by incorporating an \gls{ANN} into the \gls{SEIR} model to eliminate the compartmental model basic assumptions about the contact rate and the fatality being static.
We further tried to improve the forecast performance of the model by using mobility data and social network data as input features to the embed \gls{ANN}.
The model utilized the \gls{UDE} approach \cite{rackauckasUniversalDifferentialEquations2020} to learn from real-world data and make predictions backed by domain-specific knowledge.
We showed that the model could capture the trend for the evolution of the disease in many of the tested locations and provide epidemiological significant insights about the disease.
Our experiments with encoding mobility data and social network data as covariates into the model revealed that these data points could not improve the forecast performance of the model.
However, these results were not conclusive since we had a small sample size and the existing issues with training the model on real-world data might restrict our ability to observe the improvements brought by these covariates.

Even though the \gls{UDE} method had shown promising results on simulated data, our application of the approach on real-world Covid-19 data revealed many difficulties.
These issues were circumvented with different techniques suggested by existing researches \cite{kimStiffNeuralOrdinary2021,rackauckasUniversalDifferentialEquations2020}.
Having said that, not all issues were alleviated and many additional enhancements can be made to improve the model performance.
Firstly, a model with additional compartments can be used to represents the characteristics of Covid-19 such as asymptomatic infection, vaccinations, etc.
Secondly, additional time-varying covariates can be considered to inform the model, especially in cases when the disease dynamics change rapidly.
Thirdly, new improvements to training \glspl{UDE} should be considered to make a better guarantee of model convergence.
Finally, using the model in combination with recent advancements in methods for identifying unknown dynamical systems, such as \gls{SINDy} \cite{bruntonDiscoveringGoverningEquations2016}, can help to extract the underlying governing equation from the embed \gls{ANN} and provide a fully interpretable model.