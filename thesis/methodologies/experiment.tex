\section{Experiment}

An experiment were conducted to evaluate the performance of our model on out-of-sample data when different sets of covariates were chosen to inform the model.
For the experiment, we first trained the different versions of the model described in \autoref{sec:methodologies-models-definitions} using the parameters optimization procedure described in \autoref{sec:methodologies-parameters-optimization}.
After the model training process was done, we evaluated the performance of each version of the model on out-of-sample data for 4 different future horizons: 1-week horizon, 2-week horizon, 3-week horizon, and 4-week horizon.
During the training process, the model was trained

We used data for 2 countries Vietnam and the \gls{US} to train and evaluate the performance of our model with country-level data.
Two versions of the model were evaluated in this setting are the baseline model and the model that used Facebook's Movement Range Maps dataset.
Because the \gls{SPC} index was not defined at the country-level, the version that used the \gls{SPC} index was not considered.
Data at the country-level was selected for a period of about 2 months (60 days) where the first 32 days of the period were used to train the model and the last 28 days of the period were used to evaluate the model out-of-sample performance.
We chose the first date in the selected period to be the date where number of total confirmed cases passed 500 at each considered country.
The specific dates ranges that we used to select the data were given in \autoref{tab:chosen-dataset-dates}.
Once the data was obtained, we trained the model with the ADAM optimizer learning rate set to $1e-2$ and with the BFGS optimizer initial stepnorm set to $1e-2$.
The ADAM optimizer was let running for a maximum of 8000 iterations and the BFGS optimizer was let running for a maximum of 1000 iterations.
The $\zeta$ hyperparameter for adjusting the weighting factor for the loss function in \autoref{eq:ude-model-loss} was chosen to be $0.1$.
The initial conditions and initial parameters that were chosen for training are given in \autoref{tab:ude-model-initial-conditions} and \autoref{tab:ude-model-initial-parameters}.

At the country's subdivision-level, we used data for the top 4 most populous counties in the \gls{US}, which are Los Angeles (California), Cook County (Illinois), Harris County (Texas), and Maricopa Country (Arizona).
In addition, we used data for the 4 provinces in Vietnam that had the highest cases count during the outbreak starting from April 27h 2021, which are Ho Chi Minh city, Binh Duong, Dong Nai, and Long An.
In this setting, all 3 version of the model were trained and evaluated.
Similar to how the experiment was conducted with country-level data, we also obtained the Covid-19 cases count for a period of 60 days where the first 32 days of the period were used to train the model and the last 28 days were used to evaluate the model out-of-sample performance.
The first date in the selected period was the date where the number of total confirmed cases passed 500 at each considered location.
\autoref{tab:chosen-dataset-dates} contains a list of dates ranges that we considered when training and evaluating the model.
To train the model with country subdivisions' data, we used a learning rate of $1e-2$ for the ADAM optimizer and an initial stepnorm of $1e-2$ with the BFGS optimizer.
The ADAM optimizer was used for a maximum of 8000 iterations before the BFGS optimizer was applied for another 1000 iterations.
The $\zeta$ hyperparameters for the loss function in \autoref{eq:ude-model-loss} was set to $0.1$.
\autoref{tab:ude-model-initial-conditions} and \autoref{tab:ude-model-initial-parameters} contain the initial conditions and initial parameters that we chose during the training procedure.

In the 2 experiment settings, the choice of 500 cases were adopted from \cite{dandekarMachineLearningAidedGlobal2020a} in which the authors illustrated that the that model could not generalize well enough when training with data before the 500 cases threshold.
We noted that while the model might be sensitive to the choice of the initial conditions and the initial parameters, we did not considered

\begin{table}[h]
    \centering
    \begin{tabular}{| c | c | c | c | c | c | c | c |}
        Location & S & E & I & R & D & C & N \\
        \hline\hline
        Vietnam & 9.7581192e7 & 948 & 474 & 86 & 0 & 560 & 9.75827e7 \\
        \hline
        Ho Chi Minh city & 9.226579e6 & 508 & 254 & 254 & 5 & 513 & 9.227595e6 \\
        \hline
        Binh Duong & 2.579584e6 & 508 & 254 & 254 & 0 & 508 & 2.5806e6 \\
        \hline
        Dong Nai & 3.176393e6 & 502 & 251 & 252 & 2 & 505 & 3.177398e6 \\
        \hline
        Long An & 1.712664e6 & 514 & 257 & 258 & 7 & 522 & 1.713693e6 \\
        \hline
        United States & 3.32888343e8 & 982 & 491 & 7 & 21 & 519 & 3.32889823e8 \\
        \hline
        Los Angeles, California & 1.0038043e7 & 528 & 264 & 265 & 7 & 536 & 1.00391e7 \\
        \hline
        Cook, Illinois & 5.149141e6 & 544 & 272 & 273 & 3 & 548 & 5.15023e6 \\
        \hline
        Harris, Texas & 4.712287e6 & 512 & 256 & 257 & 13 & 526 & 4.713312e6 \\
        \hline
        Maricopa, Arizona & 4.484329e6 & 540 & 270 & 270 & 5 & 545 & 4.485409e6 \\
        \hline
    \end{tabular}
    \caption{Initial conditions at each modeled region for solving the systems of \glspl{ODE} defined by each model. For Vietnam's provinces, the initial $D$ and $C$ states were taken from the real-world data, and we assumed that initial $I$ and $D$ states each took account for half of the remaining total confirmed cases, $(C - D) / 2$. In addition, we assumed the initial $E$ state is twice as large as the initial $I$ state. The rest of the states for Vietnam's provinces were then derived by their definitions. For Vietnam country-level data, the initial states $I$, $R$, $D$, and $C$ were taken from real-world data, and we assumed that the initial $E$ state is twice as large as the initial $I$ state. The rest of the states for Vietnam country-level data were then derived by their definitions.}
    \label{tab:ude-model-initial-conditions}
\end{table}

\begin{table}[h]
    \centering
    \begin{tabular}{| c | c |}
        Parameter & Value \\
        \hline\hline
        $\gamma$ & $1/2$ \\
        \hline
        $\lambda$ & $1/14$ \\
        \hline
        $\alpha$ & $0.025$ \\
        \hline
        $\theta$ & \text{Randomly initialized} \\
        \hline
    \end{tabular}
    \caption{Initial parameters for solving the systems of \glspl{ODE} defined by each model. The values of $\gamma$, $\lambda$, and $\alpha$ were chosen to match existing information about Covid-19 where the mean incubation period was roughly 2 days, the mean infective period was roughly 14 days, and the fatality rate was roughly $2.5\%$.}
    \label{tab:ude-model-initial-parameters}
\end{table}

\begin{table}[h]
    \centering
    \begin{tabular}{| c | c | c | c |}
        Location & First train date & Last train date & Last evaluation date \\
        \hline\hline
        Vietnam & May 9th 2021 & June 9th 2021 & July 7th 2021 \\
        \hline
        Ho Chi Minh city & June 9th 2021 & July 10th 2021 & August 7th 2021 \\
        \hline
        Binh Duong & July 2nd 2021 & August 2nd 2021 & August 30th 2021 \\
        \hline
        Dong Nai & July 15th 2021 & August 15th 2021 & September 12th 2021 \\
        \hline
        Long An & July 13th 2021 & August 13th 2021 & September 10th 2021 \\
        \hline
        United States & March 8th 2020 & April 8th 2020 & May 6th 2020 \\
        \hline
        Los Angeles, California & March 23rd 2020 & April 23rd 2020 & May 21st 2020 \\
        \hline
        Cook, Illinois & March 21st 2020 & April 21st 2020 & May 19th 2020 \\
        \hline
        Harris, Texas & March 29th 2020 & April 29th 2020 & May 27th 2020 \\
        \hline
        Maricopa, Arizona & March 29th 2020 & April 29th 2020 & May 27th 2020 \\
        \hline
    \end{tabular}
    \caption{}
    \label{tab:chosen-dataset-dates}
\end{table}