\section{Experiments}
\label{sec:methodologies-experiments}

Several experiments were conducted to evaluate the performance of the model on out-of-sample data when different sets of covariates were chosen to inform the model.
For the experiments, the different versions of the model described in \autoref{sec:methodologies-models-definitions} were first trained using the parameters optimization procedure described in \autoref{sec:methodologies-parameters-estimation}.
During the training process, the model was separately trained with data from each considered location.
After the model training process was done, the performance of each version of the model was evaluated on out-of-sample data for 4 different future horizons: 1-week horizon, 2-week horizon, 3-week horizon, and 4-week horizon.
The the evaluation results of all versions of the model were compared against each other to set if the covariates can improve the model performance.

Data at both the country level and the subdivision level was collected for a period of about 3 months (76 days) where the first 48 days of the period were used to train the model and the last 28 days of the period were used to evaluate the model out-of-sample performance.
The period of 48 days was chosen to train the model because it was found that the model yielded the best predictions when trained with data for 48 days while still kept the training duration relatively low.
The others training periods that had been against with were 32 days and 60 days.
With 32 days, the model could fit well to the data from the training period and provided useful insights about the disease but it failed to extrapolate and had the low accuracy when tested on out-of-sample data.
While with 60 days, it took much longer to finish training the model but the accuracy did not significantly improves comparing to training with 48 days.
At each location, data was obtained starting from the first date where the total confirmed cases passed 500 cases and the total number of deaths passed 10 cases.
The specific date ranges that were used at each location are given in \autoref{tab:chosen-dataset-dates}.
In the experiments, The initial conditions varied across different locations depending on the ground truth data.
\autoref{tab:ude-model-initial-conditions} presents the values that were chosen as the initial conditions across different locations.
Furthermore, the initial parameters and their upper and lower bounds were initialized using the same method across different versions of the model and across different locations.
\autoref{tab:ude-model-initial-parameters} lists the values that were chosen as the initial parameters across different locations.

\begin{table}[h]
    \centering
    \begin{tabular}{| c | c | c | c |}
        Location & First train date & Last train date & Last evaluation date \\
        \hline\hline
        Vietnam & 27th April 2021 & 28th May 2021 & 25th June 2021 \\
        \hline
        Ho Chi Minh city & 9th June 2021 & 10th July 2021 & 7th August 2021 \\
        \hline
        Binh Duong & 2nd July 2021 & 2nd August 2021 & 30th August 2021 \\
        \hline
        Dong Nai & 15th July 2021 & 15th August 2021 & 12th September 2021 \\
        \hline
        Long An & 13th July 2021 & 13th August 2021 & 10th September 2021 \\
        \hline
        United States & 1st July 2021 & 1st August 2020 & 29th August 2020 \\
        \hline
        Los Angeles, California & 1st July 2021 & 1st August 2020 & 29th August 2020 \\
        \hline
        Cook, Illinois & 1st July 2021 & 1st August 2020 & 29th August 2020 \\
        \hline
        Harris, Texas & 1st July 2021 & 1st August 2020 & 29th August 2020 \\
        \hline
        Maricopa, Arizona & 1st July 2021 & 1st August 2020 & 29th August 2020 \\
        \hline
    \end{tabular}
    \caption{The date ranges that were chosen for the training process and the evaluation process of the model.}
    \label{tab:chosen-dataset-dates}
\end{table}

\begin{table}[h]
    \centering
    \begin{tabular}{| c | c | c | c | c |}
        Location & $S(0)$ & $E(0)$ & $I(0)$ & $R(0)$ \\
        \hline\hline
        Vietnam & 9.75e7 & 25 & 5 & 2817 \\
        \hline
        Ho Chi Minh city & 9.22e6 & 173.5 & 34.7 & 360.1 \\
        \hline
        Binh Duong & 2.58e6 & 206.4 & 41.2 & 314.7 \\
        \hline
        Dong Nai & 3.17e6 & 295 & 59 & 194.8 \\
        \hline
        Long An & 1.71e6 & 217.8 & 43.5 & 300.5 \\
        \hline
        United States & 2.99e8 & 71890 & 14378 & 3.31e7 \\
        \hline
        Los Angeles, California & 8.78e6 & 2500 & 500 & 1.22e6 \\
        \hline
        Cook, Illinois & 4.59e6 & 615 & 123 & 546508 \\
        \hline
        Harris, Texas & 4.30e6 & 930 & 186 & 396430 \\
        \hline
        Maricopa, Arizona & 3.92e6 & 1325 & 265 & 549899 \\
        \hline
    \end{tabular}
    \caption{Initial conditions at each modeled region for solving the systems of \glspl{ODE} defined by each model. The initial values for the states $\{D, N, C, T\}$ were taken directly from the datasets, and they are not presented in this table. The value of $I(0)$ was assumed to be the number of new cases on the date of the first time step, and $E(0)$ was assumed to be 5 times the value of $I(0)$. Then $S(0)$ was calculated by subtracting $T(0)$ and $E(0)$ from average population. Lastly, $R(0)$ was calculated by subtracting $I(0)$ and $D(0)$ from $T(0)$.}
    \label{tab:ude-model-initial-conditions}
\end{table}

\begin{table}[h]
    \centering
    \begin{tabular}{| c | c | c | c |}
        Parameter & Value & Lower bound & Upper bound \\
        \hline\hline
        $\beta$ & \textit{N/A} & $0.05$ & $1.67$ \\
        \hline
        $\gamma$ & $1/4$ & $1/4$ & $1/4$ \\
        \hline
        $\lambda$ & $1/14$ & $1/14$ & $1/14$ \\
        \hline
        $\alpha$ & \textit{N/A} & $0.005$ & $0.05$ \\
        \hline
        $\theta_1$ & \textit{glorot\_normal initialization} & \textit{N/A} & \textit{N/A} \\
        \hline
        $\theta_2$ & \textit{glorot\_normal initialization} & \textit{N/A} & \textit{N/A} \\
        \hline
    \end{tabular}
    \caption{Initial parameters for solving the systems of \glspl{ODE} defined by each model. The values of $\gamma$, $\lambda$, and $\alpha$ were chosen to match existing information about the Covid-19 Delta variant \cite{mahaseDeltaVariantWhat2021} where the mean incubation period was roughly 4 days, and the mean infective period was roughly 14 days.}
    \label{tab:ude-model-initial-parameters}
\end{table}

At the country level, data for 2 countries Vietnam and the \gls{US} were used to train and evaluate the performance of the model.
Two versions of the model evaluated in this setting were the baseline model and the model that used Facebook's Movement Range Maps dataset.
Because the \gls{SPC} index was not defined at the country level, the version that used the \gls{SPC} index was not considered.
At the country's subdivision level, all three versions of the model were trained on Covid-19 time series data for the top 4 most populous counties in the \gls{US} (Los Angeles - California, Cook County - Illinois, Harris County - Texas, and Maricopa County - Arizona) and the top 4 provinces in Vietnam that had the highest cases count during the outbreak starting from 27th April 2021 (Ho Chi Minh city, Binh Duong, Dong Nai, and Long An).

Once the data was obtained, the model was first trained using ADAM optimizer with learning rate set to $0.05$, and the learning rate was exponentially decayed with the rate of $0.5$ for each $1000$ iterations until the minimum value of $0.00001$ was reached.
Using iterative growing of fit strategy, the model was first trained on a time span of size 4 which was then increased by 4 after each training session on the previous fitting time span finished.
At each time span in the first stage, the ADAM optimizer was configured to run for the maximum of 1000 iterations.
When the first training stage finished, BFGS optimizer was used for a maximum of $1000$ iterations with the \textit{initial\_stepnorm} parameter set to $0.01$
The hyperparameter $\zeta$ for adjusting the weighting factor for the loss function in \autoref{eq:ude-model-loss} was chosen to be $-0.001$ to place more importance on earlier time steps which minimized the chance of getting into a bad local minima.
Finally, the hyperparameter $\lambda$ was set to $0.00001$ to help reduce the change of over-fitting.

Noted that while the model might be sensitive to the choice of the initial conditions and the initial parameters, the effects of choosing a different set of initial conditions and initial parameters were not considered.
The Covid-19 dataset, the Movement Range Maps dataset, and the derived \gls{SPC} index all exhibited weekly seasonality in the data.
In the Movement Range Maps dataset, the relative change in movement tended to drop to a much lower value while the stay-put ratio increased sharply in the weekend.
This was expected as the data was based on human mobility where people went to work during the week and stayed rested on the weekend.
With the Covid-19 cases data the \gls{SPC} index, the value typically dipped in value in the weekend when fewer Covid-19 tests were made.
Therefore, in the experiments, a 7-day moving average transform was applied to the Covid-19 cases data, the Movement Range Maps dataset, and the \gls{SPC} index to remove the weekly seasonality exhibited in the measurements.
The elimination of weekly seasonality in the data allowed the model to better capture the overall trends and helped to avoid confusing the model with highly fluctuating values.
Moreover, after being transformed with a 7-day moving average, values from the Movement Range Maps dataset and the \gls{SPC} index were scaled to have values ranging from 0 to 1 within the training time span.
This ensured all the input features to the \glspl{ANN} had the same scale during training which would help improve the performance of the model.