\section{Experiments}
\label{sec:methodologies-experiments}

Several experiments were conducted to evaluate the performance of the model on out-of-sample data when different sets of covariates were chosen to inform the model.
For the experiments, the different versions of the model described in \autoref{sec:methodologies-models-definitions} were first trained using the parameters optimization procedure described in \autoref{sec:methodologies-parameters-estimation}.
During the training process, the model was separately trained with data from each considered location.
After the model training process was done, the performance of each version of the model was evaluated on out-of-sample data for 4 different future horizons: 1-week horizon, 2-week horizon, 3-week horizon, and 4-week horizon.
The the evaluation results of all versions of the model were compared against each other to set if the covariates can improve the model performance.

Data at both the country level and the subdivision level was collected for a period of about 2 months (60 days) where the first 32 days of the period were used to train the model and the last 28 days of the period were used to evaluate the model out-of-sample performance.
At each location, data was obtained starting from the first date where the total confirmed cases passed 500.
The specific date ranges that were used at each location are given in \autoref{tab:chosen-dataset-dates}.
The initial conditions and initial parameters that were chosen for training are given in \autoref{tab:ude-model-initial-conditions} and \autoref{tab:ude-model-initial-parameters}.
In the experiments, the initial parameters constant are kept constant across different versions of the model and across different locations.
In contrast, the initial conditions varied across different locations depending on the ground truth data.

\begin{table}[h]
    \centering
    \begin{tabular}{| c | c | c | c |}
        Location & First train date & Last train date & Last evaluation date \\
        \hline\hline
        Vietnam & 9th May 2021 & 9th June 2021 & 7th July 2021 \\
        \hline
        Ho Chi Minh city & 9th June 2021 & 10th July 2021 & 7th August 2021 \\
        \hline
        Binh Duong & 2nd July 2021 & 2nd August 2021 & 30th August 2021 \\
        \hline
        Dong Nai & 15th July 2021 & 15th August 2021 & 12th September 2021 \\
        \hline
        Long An & 13th July 2021 & 13th August 2021 & 10th September 2021 \\
        \hline
        United States & 8th March 2020 & 8th April 2020 & 6th May 2020 \\
        \hline
        Los Angeles, California & 23rd March 2020 & 23rd April 2020 & 21st May 2020 \\
        \hline
        Cook, Illinois & 21st March 2020 & 21st April 2020 & 19th May 2020 \\
        \hline
        Harris, Texas & 29th March 2020 & 29th April 2020 & 27th May 2020 \\
        \hline
        Maricopa, Arizona & 29th March 2020 & 29th April 2020 & 27th May 2020 \\
        \hline
    \end{tabular}
    \caption{The date ranges that were chosen for the training process and the evaluation process of the model.}
    \label{tab:chosen-dataset-dates}
\end{table}

\begin{table}[h]
    \centering
    \begin{tabular}{| c | c | c | c | c | c | c | c |}
        Location & S & E & I & R & D & C & N \\
        \hline\hline
        Vietnam & 9.7581192e7 & 948 & 474 & 86 & 0 & 560 & 9.75827e7 \\
        \hline
        Ho Chi Minh city & 9.226579e6 & 508 & 254 & 254 & 5 & 513 & 9.227595e6 \\
        \hline
        Binh Duong & 2.579584e6 & 508 & 254 & 254 & 0 & 508 & 2.5806e6 \\
        \hline
        Dong Nai & 3.176393e6 & 502 & 251 & 252 & 2 & 505 & 3.177398e6 \\
        \hline
        Long An & 1.712664e6 & 514 & 257 & 258 & 7 & 522 & 1.713693e6 \\
        \hline
        United States & 3.32888343e8 & 982 & 491 & 7 & 21 & 519 & 3.32889823e8 \\
        \hline
        Los Angeles, California & 1.0038043e7 & 528 & 264 & 265 & 7 & 536 & 1.00391e7 \\
        \hline
        Cook, Illinois & 5.149141e6 & 544 & 272 & 273 & 3 & 548 & 5.15023e6 \\
        \hline
        Harris, Texas & 4.712287e6 & 512 & 256 & 257 & 13 & 526 & 4.713312e6 \\
        \hline
        Maricopa, Arizona & 4.484329e6 & 540 & 270 & 270 & 5 & 545 & 4.485409e6 \\
        \hline
    \end{tabular}
    \caption{Initial conditions at each modeled region for solving the systems of \glspl{ODE} defined by each model. For Vietnam's provinces and \gls{US} counties, the initial $D$ and $C$ states were taken from the real-world data. Each of the initial $I$ and $D$ states was assumed to take account for half of the remaining total confirmed cases, $(C - D) / 2$. In addition, the initial $E$ state was assumed to be twice as large as the initial $I$ state. The rest of the states for were then derived by their definitions. For Vietnam and the \gls{US} country-level data, the initial states $I$, $R$, $D$, and $C$ were taken from real-world data, and the initial $E$ state was assumed to be twice as large as the initial $I$ state. The rest of the states for country-level data were then derived by their definitions.}
    \label{tab:ude-model-initial-conditions}
\end{table}

\begin{table}[h]
    \centering
    \begin{tabular}{| c | c |}
        Parameter & Value \\
        \hline\hline
        $\gamma$ & $1/2$ \\
        \hline
        $\lambda$ & $1/14$ \\
        \hline
        $\alpha$ & $0.025$ \\
        \hline
        $\theta$ & \text{Randomly initialized} \\
        \hline
    \end{tabular}
    \caption{Initial parameters for solving the systems of \glspl{ODE} defined by each model. The values of $\gamma$, $\lambda$, and $\alpha$ were chosen to match existing information about the Covid-19 Delta variant \cite{mahaseDeltaVariantWhat2021} where the mean incubation period was roughly 2 days, the mean infective period was roughly 14 days, and the fatality rate was roughly $2.5\%$.}
    \label{tab:ude-model-initial-parameters}
\end{table}

At the country level, data for 2 countries Vietnam and the \gls{US} were used to train and evaluate the performance of the model.
Two versions of the model were evaluated in this setting are the baseline model and the model that used Facebook's Movement Range Maps dataset.
Because the \gls{SPC} index was not defined at the country level, the version that used the \gls{SPC} index was not considered.
Once the data was obtained, the model was trained with the ADAM optimizer learning rate set to $0.01$ and with the BFGS optimizer initial stepnorm set to $0.01$.
The ADAM optimizer was let run for a maximum of 500 iterations and the BFGS optimizer was let run for a maximum of 1000 iterations.
The hyperparameter $\zeta$ for adjusting the weighting factor for the loss function in \autoref{eq:ude-model-loss} was chosen to be $0.1$.

At the country's subdivision level, data for the top 4 most populous counties in the \gls{US} were , which are Los Angeles (California), Cook County (Illinois), Harris County (Texas), and Maricopa Country (Arizona).
In addition, data for the 4 provinces in Vietnam that had the highest cases count during the outbreak starting from 27th April 2021 were also used, which are Ho Chi Minh city, Binh Duong, Dong Nai, and Long An.
In this setting, all three versions of the model were trained and evaluated.
To train the model with country subdivisions' data, a learning rate of $0.01$ for the ADAM optimizer and an initial stepnorm of $0.01$ for the BFGS optimizer were used.
The ADAM optimizer was used for a maximum of 500 iterations before the BFGS optimizer was applied for another maximum of 1000 iterations.
The hyperparameter $\zeta$ for the loss function in \autoref{eq:ude-model-loss} was set to $0.1$.

In the 2 experiment settings, the choice of 500 cases was adopted from \cite{dandekarMachineLearningAidedGlobal2020a} in which the authors illustrated that the model could not generalize well enough when training with data before the 500 cases threshold.
Noted that while the model might be sensitive to the choice of the initial conditions and the initial parameters, the effects of choosing a different set of initial conditions and initial parameters were not considered.
The Covid-19 dataset, the Movement Range Maps dataset, and the derived \gls{SPC} index all exhibited weekly seasonality in the data.
In the Movement Range Maps dataset, the relative change in movement tended to drop to a much lower value while the stay-put ratio increased sharply in the weekend.
This was expected as the data was based on human mobility where people went to work during the week and stayed rested on the weekend.
With the Covid-19 cases data the \gls{SPC} index, the value typically dipped in value in the weekend when fewer Covid-19 tests were made.
In the experiments, a 7-day moving average transform was applied to the Covid-19 cases data, the Movement Range Maps dataset, and the \gls{SPC} index to remove the weekly seasonality exhibited in the measurements.
The elimination of weekly seasonality in the data allowed the model to better capture the overall trends and helped to avoid confusing the model with highly fluctuating values.