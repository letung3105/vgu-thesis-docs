\section{Experiments}

Several experiments were conducted to evaluate the performance of our model on out-of-sample data when different sets of covariates were chosen to inform the model.
For the experiments, we first trained the different versions of the model described in \autoref{sec:methodologies-models-definitions} using data for the top 4 counties in the \gls{US} with the most confirmed cases, data for the top 4 provinces in Vietnam with the most confirmed cases, and country-level data for Vietnam.
Each version of the model was trained to find the best fitting parameters using solely the data for the region that was being considered.
Once we had the trained model for a specific region, it was then used to simulate the subsequent dates after the training period for 4 different forecasting horizons: 1-week horizon, 2-week horizon, 3-week horizon, and 4-week horizon.
Details about the data used in both the training period and testing period, and how they were processed were given in \autoref{sec:methodologies-data}
The outputs for the 4 forecasting horizons from all versions of the model were when evaluated and compared against each other using multiple different metrics defined in \autoref{sec:methodologies-evaluation-metrics}.
Because the \gls{SPC} index \cite{kuchlerGeographicSpreadCOVID192020} only informed about the inter-province/inter-province spread of the disease, the training and evaluation process that used Vietnam country-level data did not consider the version of the model that encoded the \gls{SPC} index.