\section{Models definitions}
\label{sec:methodologies-models-definitions}

As mentioned in previous sections, we were utilizing a compartmental model for infectious diseases as the basis for our model.
We chose the basic \gls{SEIR} model because of its simplicity and a low number of parameters which resulted in a simpler and more manageable learning process.
We chose not to include demographic effects in the model because we were considering a short outbreak period, thus these effects were negligible.
Moreover, we did not include additional compartments for individuals that have been vaccinated because the vaccination was only available for a low percentage of the population for the period that we chose to apply our model.
In addition to the existing compartments in the \gls{SEIR} model, we introduced two additional compartments $D$ and $C$ for the number of total deaths and the number of total confirmed cases.
The two compartments $D$ and $C$ were introduced so that the model could represent the available observations that we had with real-world data.
Formally, the system of \glspl{ODE} that governs our extended \gls{SEIR} model was defined as
\begin{equation}
    \begin{aligned}
        S' &= - \frac{\beta SI}{N} \\
        E' &= \frac{\beta SI}{N} - \gamma E \\
        I' &= \gamma E - \lambda I \\
        R' &= (1 - \alpha) * \lambda * I \\
        D' &= \alpha * \lambda * I \\
        C' &= \gamma * E \\
        N' &= - \alpha * \lambda * I
    \end{aligned}
    \label{eq:methodologies-seir-model}
\end{equation}
where $\beta$ is the infection rate, $\gamma$ is the exposure rate, $\lambda$ is the rate at which infective individuals become not infective anymore, and $\alpha$ is the fatality rate.
Unlike the definition of the \gls{SEIR} model given in \autoref{sec:literature-review-compartmental-seird-model} where an infective person can infect $\beta N$ other people per unit time, here we let $\beta$ itself be the number of people that can be infected by a single infective individual per unit time.
Thus, the total number of newly infected people per unit time in this formulation turned into $\beta SI / N$ instead of $\beta SI$.
This was done because, in our experiment, we observed that using $\beta$ as the number of people that can be infected by a single infective individual resulted in a better fit to the training data and a better out-of-sample performance of the model.
We noted that existing literature \cite{arikInterpretableSequenceLearning, bastosModelingForecastingEarly2020,dengDynamicsDevelopmentCOVID192020, heSEIRModelingCOVID192020,ihmecovid-19forecastingteamModelingCOVID19Scenarios2021, ndairouMathematicalModelingCOVID192020,sarkarModelingForecastingCOVID192020, zhaoModelingEpidemicDynamics2020} had also utilized the same formulation for the transition between the $S$ and $E$ compartments.
Based on \autoref{eq:methodologies-seir-model}, we could obtain the basic reproduction number by using the following equation
\begin{equation*}
    \mathcal{R}_0 = \frac{\beta}{\gamma}.
    \label{eq:methodologies-seir-reproduction-number}
\end{equation*}

Because it had been shown that several factors could have varying effects on the transmission rate $\beta$, and a time-dependent transmission rate $\beta(t)$ was much more representative of Covid-19 \cite{arikInterpretableSequenceLearning, changMobilityNetworkModels2021, dandekarMachineLearningAidedGlobal2020a, ihmecovid-19forecastingteamModelingCOVID19Scenarios2021, liSubstantialUndocumentedInfection2020}, we incorporated an \gls{ANN} that could encode different covariates into the system of equations so that the model could represent a time-dependent contact rate $\beta(t)$.
Generally, we have
\begin{equation}
    \beta(t) = \mathcal{NN}_\theta (\mathcal{F}),
    \label{eq:methodologies-seir-time-dependent-contact-rate}
\end{equation}
where $\mathcal{NN}_\theta$ is an \gls{ANN} whose weights and biases are represented as $\theta$, and $\mathcal{F}$ is the set of covariates that were used to inform the \gls{ANN} about different effects that external factors had time-dependent contact rate $\beta(t)$.
Hence, the basic system of equations in \autoref{eq:methodologies-seir-model} was reformulated as
\begin{equation}
    \begin{aligned}
        S' &= - \frac{\beta(t) SI}{N} = -\frac{\mathcal{NN}_\theta(\mathcal{F}) SI}{N} \\
        E' &= \frac{\beta(t) SI}{N} - \gamma E = \frac{\mathcal{NN}_\theta(\mathcal{F}) SI}{N} - \gamma E \\
        I' &= \gamma E - \lambda I \\
        R' &= (1 - \alpha) * \lambda * I \\
        D' &= \alpha * \lambda * I \\
        C' &= \gamma * E \\
        N' &= - \alpha * \lambda * I
    \end{aligned}
    \label{eq:methodologies-seir-ude-model}
\end{equation}
The system of equations in \autoref{eq:methodologies-seir-ude-model} gave us an \gls{UDE} \cite{rackauckasUniversalDifferentialEquations2020} that can be trained in an end-to-end fashion similar to the training process for \glspl{ANN}.
From \autoref{eq:methodologies-seir-reproduction-number} and \autoref{eq:methodologies-seir-time-dependent-contact-rate}, we derived the basic reproduction rate of the disease at time $t$ to be
\begin{equation*}
    \mathcal{R}_t = \frac{\beta(t)}{\gamma} = \frac{\mathcal{NN}_\theta(\mathcal{F})}{\gamma}.
\end{equation*}

In \cite{rackauckasUniversalDifferentialEquations2020}, the authors show that a small \gls{ANN} with a low number of hidden layers and a low number of nodes is sufficient for an \gls{UDE} to represent the complex interactions in different dynamical systems.
A similar model proposed in \cite{dandekarMachineLearningAidedGlobal2020a} had also used a small \gls{ANN} with 2 hidden layers where each layer had 10 nodes, and it had been shown to achieve a good performance on Covid-19 data with that small \gls{ANN}.
As a result, the $\mathcal{NN}_\theta$ embedded in \autoref{eq:methodologies-seir-ude-model} was chosen to be a small \gls{ANN} with 3 hidden layers where each layer had 8 nodes.
The \gls{ReLU} function was used as the activation function for all hidden layers, and the softplus function was used as the activation function for the output layer.
We applied the softplus function to the output layer so that the constraint on $\beta(t)$ to be a positive number was satisfied.
Besides using \gls{ReLU} as the activation function for the hidden layers, we also tried other activation functions, namely \textit{tanh} and \textit{sigmoid}.
But because \gls{ReLU} gave the best performance out of all the activation functions that we had tested in the preliminary trials, it was chosen as the activation function for our general \gls{UDE} model in \autoref{eq:methodologies-seir-ude-model}.
We noted that a different \gls{ANN}'s architecture might improve the performance of our model, but we decided not to explore the vast number of the different \gls{ANN} architectures that could be employed.

Using the \gls{UDE} model in the general form given by \autoref{eq:methodologies-seir-ude-model}, we then explored different sets of external factors that could have an effect on the time-dependent contact rate $\beta (t)$ to see which covariates can improve the performance of the model.
When we considered the different sets of covariates as inputs to the \gls{ANN}, the only parameters of the \gls{ANN} that change is the number of input features that can be taken by the \gls{ANN}, while the number of hidden layers, the number of nodes, and the activation function of each layer stayed the same.
In the subsequent sections, we describe the different sets of covariates that were chosen as input the $\mathcal{NN}_\theta$ in \autoref{eq:methodologies-seir-ude-model}.

\subsubsection{Informing the contact rate with past states}
The first set of covariates that we utilized as inputs for the \gls{ANN} in \autoref{eq:methodologies-seir-ude-model} was the states of the system in the previous time step.
We assumed that two states that could have an effect on the contact rate $\beta$ were the number of infective individuals $I$ and the fraction of susceptible individuals among the population $S/N$.
In other words, the set of covariates $\mathcal{F}$ that were given as input to the \gls{ANN} at time step $t$ in this model was defined as
\begin{equation*}
    \mathcal{F}(t) = \left\lbrace \frac{S(t-1)}{N(t-1)}, \frac{I(t-1)}{N(t-1)} \right\rbrace
\end{equation*}
In this form, our model is similar to the one that had been proposed in \cite{dandekarMachineLearningAidedGlobal2020a}.
The other states of the system were not considered as inputs to the \gls{ANN} in this version of the model because, by definition, the individuals from those states can not transmit the disease.
Thus they should have zero effect on the transmission rate $\beta(t)$.
Furthermore, we consider the fractions $S/N$ and $I/N$ as inputs instead of the actual counts because from our preliminary trials with the model, we observed that it helped to create better results and a faster runtime.
Moreover, using $S/N$ and $I/N$ also partially encoded the effective population size $N$ which better represented the prevalence of the disease.
This model did not rely on external data to inform the embedded \gls{ANN} and based its prediction entirely on the states of the system.
As such, the model was used as the baseline for comparisons in the experiments.

\subsubsection{Informing the model with mobility data}

As noted by the authors of \cite{dandekarMachineLearningAidedGlobal2020a}, informing the basic \gls{UDE} model for Covid-19 with mobility data could improve the forecasting ability of the model.
Moreover, \cite{changMobilityNetworkModels2021,ihmecovid-19forecastingteamModelingCOVID19Scenarios2021,liSubstantialUndocumentedInfection2020} have showed that mobility is an important predictor for the spread of Covid-19.
Therefore, we used the Movement Range Maps dataset from Facebook (see \autoref{sec:methodologies-data-mobility-data}) in addition to the past system's states in the second set of covariates that were given as inputs for the \gls{ANN} in \autoref{eq:methodologies-seir-ude-model}.
In this second version of the model, the set of covariates $\mathcal{F}$ given to the \gls{ANN} at time step $t$ was defined as
\begin{equation*}
    \mathcal{F}(t) = \left\lbrace \frac{S(t-1)}{N(t-1)}, \frac{I(t-1)}{N(t-1)}, \text{MovementRange}(t) \right\rbrace,
\end{equation*}
where $\text{MovementRange}(t)$ was the relative change in movement and the ratio of people who were staying put at the area that was being modeled at time $t$.
When we modeled with data for Vietnam and its provinces, $\text{MovementRange}(t)$ was calculated as the mean of the values of all the subregions within the area that was being modeled.
That was done because the Movement Range Maps dataset was containing data at a much more granular level but we were only interested in the country-level data and province-level data.

\subsubsection{Informing the model with social network connections}

As shown by the authors in \cite{kuchlerGeographicSpreadCOVID192020}, the strengths of social network connections between different regions were an important predictor of Covid-19 cases in US counties.
They had defined the \gls{SPC} index which is a metric defined specifically for Covid-19 that measures the spreads of the disease based on the \gls{SCI} dataset from Facebook.
They also noted that the correlation between the \gls{SPC} index and the number of cases was especially strong when there were fewer restrictions on individuals' mobility.
Thus, we considered the \gls{SPC} index as an input feature to the \gls{ANN} in addition to the Movement Range Maps dataset from Facebook and the past system's states.
The \gls{SPC} index is calculated as
\begin{equation*}
    SPC_{i,t} = \sum_j^C \text{Cases per 10k}_{j,t} \frac{SCI_{i,j}}{\sum_h^C SCI_{i,h}},
\end{equation*}
where $SPC_{i,t}$ is the \gls{SPC} index of location $i$ at time $t$, $C$ is the number of locations, and $\text{Cases per 10k}_{j,t}$ if the number of confirmed cases out of 10,000 people at location $j$ at time $t$.
In this third version of the model, the set of covariates $\mathcal{F}$ given to the \gls{ANN} at time step $t$ was defined as
\begin{equation*}
    \mathcal{F}(t) = \left\lbrace \frac{S(t-1)}{N(t-1)}, \frac{I(t-1)}{N(t-1)}, \text{MovementRange}(t), \text{SPC}(t - \tau) \right\rbrace,
\end{equation*}
where $\text{SPC}(t - \tau)$ was the \gls{SPC} index of the area that was being modeled at time $t - \tau$.
Because the changes in the \gls{SPC} index in one day would not be immediately reflected in the changes in the number of cases for that same day, a temporal lag $\tau$ was considered when the \gls{SPC} index was given as input to the \gls{ANN}.
In \cite{kuchlerGeographicSpreadCOVID192020}, the authors had shown that \gls{SPC} index in one 2-week period highly correlated with the growth in the number of cases in the next 2-week period.
Therefore, we chose $\tau$ to be 14 days when giving the \gls{SPC} index as input to the model.