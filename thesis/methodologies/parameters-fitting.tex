\section{Parameters fitting}
\label{sec:methodologies-parameters-fitting}

We used an end-to-end learning mechanism to find the set of model parameters that produced that best fitting curves to the ground truth data.
In the model learning period, we only have access to observations for some of the compartments in the model because of the availability of data.
With county-level data and province-level data, we did not have access to the states $\{S, E, I, R\}$ but only the $D$ and $C$ states, which are the number of total deaths and the number of total confirmed cases.
With country-level data, we did not have access to the states $\{S, E\}$ but only the $I$, $R$, $D$, and $C$ states, which are the number of infective individuals, the number of total recoveries, the number of total deaths, and the number of total confirmed cases.
Therefore, model could only learn from partially-available observations where the available data for some compartments were used to supervise the learning process while other compartments were left unconstrained.

All versions of the model were trained by minimizing the following loss function
\begin{equation}
    \mathcal{L}(\gamma, \lambda, \alpha, \theta) = \sum_{i=1}^N \sum_{t=0}^{T-1} e^{t\zeta} (log(\hat{y}_{i,t} + 1) - log(y_{i,t} + 1))^2.
    \label{eq:ude-model-loss}
\end{equation}
$N$ is the number of compartments that are observable, $T$ is the number of collocation points, i.e., the number of days that were used for training, and $e^{t\zeta}$ is a weighting factor that places higher priority to later data points.
Lastly, $\hat{y}_{t,t}$ is the model's output for compartment $i$ at time $t$, and $y_{i,t}$ is the observation for compartment $i$ at time $t$.
The errors between the ground truth data and our model's predictions were taken in logarithmic scale, $log(\hat{y}_{i,t} + 1) - log(y_{i,t} + 1)$, because the states of the system were in vastly different scales.
Here, the number of total confirmed cases were much larger than the number of deaths or the number of recoveries, which dominated the loss value and made the model learned poorly.
Before taking the log, we added $1$ to the values to take into account when the ground truth values or the predicted values were $0$.

Before we could minimizer the loss function in \autoref{eq:ude-model-loss}, we must first solve the system of \glspl{ODE} to obtain the model's outputs.
We used a numerical \gls{ODE} solver using the Tsitouras 5/4 Runge-Kutta method \cite{tsitourasRungeKuttaPairs2011} to solve our model's system of \glspl{ODE}.
Once we obtained the model's outputs, minimization of the loss function in \autoref{eq:ude-model-loss} was carried out with adjoint sensitivity analysis \cite{maComparisonAutomaticDifferentiation2021} using the approach described by \citeauthor{rackauckasUniversalDifferentialEquations2020} \cite{rackauckasUniversalDifferentialEquations2020}.
\autoref{tab:ude-model-initial-conditions} contains a list of all the values chosen as the initial conditions for solving the systems of \glspl{ODE}, and \autoref{tab:ude-model-initial-parameters} contains a list of all the values chosen as the initial parameters for solving the systems of \glspl{ODE}.

\begin{table}[h]
    \centering
    \begin{tabular}{| c | c | c | c | c | c | c | c |}
        Location & S & E & I & R & D & C & N \\
        \hline\hline
        Vietnam \\
        \hline
        Ho Chi Minh city \\
        \hline
        Binh Duong \\
        \hline
        Dong Nai \\
        \hline
        Long An \\
        \hline
    \end{tabular}
    \caption{Initial conditions at each modeled region for solving the systems of \glspl{ODE} defined by each model. For Vietnam's provinces, the initial $D$ and $C$ states were taken from the real-world data, and we assumed that initial $I$ and $D$ states each took account for half of the remaining total confirmed cases, $(C - D) / 2$. In addition, we assumed the initial $E$ state is twice as large as the initial $I$ state. The rest of the states for Vietnam's provinces were then derived by their definitions. For Vietnam country-level data, the initial states $I$, $R$, $D$, and $C$ were taken from real-world data, and we assumed that the initial $E$ state is twice as large as the initial $I$ state. The rest of the states for Vietnam country-level data were then derived by their definitions.}
    \label{tab:ude-model-initial-conditions}
\end{table}

\begin{table}[h]
    \centering
    \begin{tabular}{| c | c |}
        Parameter & Value \\
        \hline\hline
        $\gamma$ & $1/2$ \\
        \hline
        $\lambda$ & $1/14$ \\
        \hline
        $\alpha$ & $0.025$ \\
        \hline
        $\theta$ & \text{Randomly initialized} \\
        \hline
    \end{tabular}
    \caption{Initial parameters for solving the systems of \glspl{ODE} defined by each model. The values of $\gamma$, $\lambda$, and $\alpha$ were chosen to match existing information about Covid-19 where the mean incubation period was roughly 2 days, the mean infective period was roughly 14 days, and the fatality rate was roughly $2.5\%$.}
    \label{tab:ude-model-initial-parameters}
\end{table}