\chapter{Discussion}
\label{chap:discussion}

Regarding \autoref{fig:fit-country-level}, \autoref{fig:fit-us-counties}, \autoref{fig:fit-vn-provinces}, and the data presented in \autoref{chap:results}, it is observed that the model could fit closely to data for the training period and provided useful insight into the evolution of the disease, which could not be done by a basic \gls{SEIR} model \cite{dandekarMachineLearningAidedGlobal2020a}.
However, the model could not fit well to data where high fluctuations were presented, which can be seen in the model's fit for Harris (Texas), Dong Nai, Long An, and Binh Duong.
Even though we had tried to eliminate high frequency changes in the data by using a 7-day moving average, high variations in the number of new cases from day to day still existed due to various reasons such as how the data was collected, the present of asymptomatic cases, underreporting, the number of tests that was performed, etc.
Moreover, while we did not study the effects that the covariates had on the model's parameters in the second and third versions, an in-depth analysis could be done to gain additional insights into which factors played an important role in slowing down to spread of the virus.

When extrapolated beyond the training period, the model had low errors on the out-of-sample data for most of the considered locations and showed that it can capture the evolution of the disease.
In instances where the model could forecasts the trend of the disease, it was observed that the model had the highest accuracy when making short-term forecasts, and the predictions accuracy deteriorated as the forecast horizon expanded.
This was expected as many real-world phenomenons could invalidate the assumptions made by the model during training, and the further we extrapolated from the training period, the more the errors resulted from these assumptions accumulated.
Two cases where the model completely failed to forecast the trend of the disease were with data for Ho Chi Minh city and Vietnam.
In these two instances, we can see that the data used for training the model, illustrated by \autoref{fig:fit-country-level} and \autoref{fig:fit-vn-provinces}, indicated a monotonically decreasing trend in the spread of the disease before the sharp increase was presented in the testing data.
Because this sudden change in the disease dynamics was not exhibited in the training data for both cases, the model failed to make an accurate forecast of when such changes would happen.
This result is also aligned with existing studies \cite{arikInterpretableSequenceLearning}.
Furthermore, in can be seen that the predictions made for locations where high fluctuations were presented were not as good as the predictions made for locations where the numbers followed smoother curves.
This result indicated the model's shortcomings in learning and generalizing from noisy data.

Considering the benefits of encoding mobility data and social network data into the model, our results demonstrated that there was no improvement in the model accuracy for both short-term and long-term forecasts when these covariates were encoded.
This finding contrasted existing findings on mobility data and \gls{SPC} index being important predictors for the future number of cases \cite{changMobilityNetworkModels2021,kuchlerGeographicSpreadCOVID192020,arikInterpretableSequenceLearning,liSubstantialUndocumentedInfection2020}.
We hypothesized several reasons to explain this discrepancy.
Firstly, only a short time span of 48 days were used to train the model which may prevent the embed \gls{ANN} from generalizing beyond the training data.
Secondly, the mobility data that we used was not as granular and complex as those used in existing researches because we only considered datasets that were available for Vietnam.
Thirdly, even though the \gls{SPC} had been shown to have high correlation with the early number of cases, it became a less important predictor when the disease prolonged and when heavy government restrictions on mobility were in place.
Finally, the underlying \gls{SEIR} model was too simple and could have introduced strong model bias that hindered the ability to learn the true interactions between the covariates and the spread of the disease.

\section{Model's convergence and generalizability}