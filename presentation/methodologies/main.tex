\section{Methodologies}

\subsection{Model definition}

\begin{frame}{Universal differential equation}
    \gls{UDE} \cite{rackauckasUniversalDifferentialEquations2020} is a neural network architecture based on \gls{NeuralODE} \cite{chenNeuralOrdinaryDifferential2019}
    \begin{block}<2->{Generalized \gls{NeuralODE}}
        \begin{equation*}
            u' = U_\theta(u, t)
        \end{equation*}
    \end{block}
    \begin{block}<3->{Generalized \gls{UDE}}
        \begin{equation*}
            u' = f(u, t, U_\theta(u, t))
        \end{equation*}
    \end{block}
\end{frame}

\begin{frame}{System of ODEs}
    \begin{columns}
        \begin{column}{0.5\linewidth}
            \begin{definition}<1->
                \begin{equation*}
                    \begin{aligned}
                        S' &= - \frac{\beta(t) SI}{N} \\
                        E' &= \frac{\beta(t) SI}{N} - \gamma E \\
                        I' &= \gamma E - \lambda I \\
                        R' &= (1 - \alpha(t)) \lambda I \\
                        D' &= \alpha(t) \lambda I \\
                        N' &= - \alpha \lambda I \\
                        C' &= -C + \gamma E \\
                        T' &= \gamma E \\
                    \end{aligned}
                \end{equation*}
            \end{definition}
        \end{column}
        \begin{column}{0.5\linewidth}
            \begin{block}<2->{Reproduction number}
                \begin{equation*}
                    \mathcal{R}^* = \frac{\beta(t)}{\gamma}
                \end{equation*}
            \end{block}
        \end{column}
    \end{columns}
\end{frame}

\begin{frame}{States graph}
    \begin{figure}
        \centering
        \begin{tikzpicture}[->, > = stealth', node distance = 2cm, thick]
            \node[state]               (S) {S};
            \node[state, right of = S] (E) {E};
            \node[state, right of = E] (I) {I};
            \node[state, above right of = E] (C) {C};
            \node[state, below right of = E] (T) {T};
            \node[state, above right of = I, xshift=1.5cm] (R) {R};
            \node[state, below right of = I, xshift=1.5cm] (D) {D};
            \draw (S) edge node[above]{$\beta(t)$} (E)
                (E) edge node[above]{$\gamma$} (I)
                (E) edge node[above,rotate=45]{$\gamma$} (C)
                (E) edge node[above,rotate=-45]{$\gamma$} (T)
                (I) edge node[above,rotate=29]{$(1 - \alpha(t)) \lambda$} (R)
                (I) edge node[above,rotate=-29]{$\alpha(t) \lambda$} (D);
        \end{tikzpicture}
        \caption{States graph for the proposed model}
    \end{figure}
\end{frame}

\begin{frame}{System parameters}
    \begin{block}<1->{Box constraints for time-independent parameters}
        \begin{equation*}
            \begin{aligned}
                \gamma &= \gamma_L + (\gamma_U - \gamma_L) * \sigma (\gamma')\\
                \lambda &= \lambda_L + (\lambda_U - \lambda_L) * \sigma (\lambda')
            \end{aligned}
        \end{equation*}
    \end{block}

    \begin{block}<2->{Neural networks for time-dependent parameters}
        \begin{equation*}
            \begin{aligned}
                \beta(t) &= \mathcal{NN}_{\theta_1}(\mathcal{F}) \\
                \alpha(t) &= \mathcal{NN}_{\theta_2} (\frac{t}{t_\text{max}}, \frac{I(t-1)}{N(t-1)}, \frac{R(t-1)}{N(t-1)}, \frac{D(t-1)}{N(t-1)})
            \end{aligned}
        \end{equation*}
        \begin{itemize}
            \item Hidden layers activation: mish \cite{misraMishSelfRegularized2020}
            \item Output layer activation $\nu_i = \nu_{i,L} + (\nu_{i,U} - \nu_{i,L}) * \sigma (z_i)$
        \end{itemize}
    \end{block}
\end{frame}

\begin{frame}{Neural network input features}
    \begin{block}<1->{1st. set of input features}
        \begin{equation*}
            \begin{aligned}
                \mathcal{F}_1(t) = \left\lbrace \frac{t}{t_\text{max}}, \frac{S(t-1)}{N(t-1)}, \frac{E(t-1)}{N(t-1)}, \frac{I(t-1)}{N(t-1)} \right\rbrace
            \end{aligned}
        \end{equation*}
    \end{block}

    \begin{block}<2->{2nd. set of input features}
        \begin{equation*}
            \mathcal{F}_2(t) = \mathcal{F}_1(t) \cup \left\lbrace \text{MovementRange}(t) \right\rbrace
        \end{equation*}
    \end{block}

    \begin{block}<3->{3rd. set of input features}
        \begin{equation*}
            \mathcal{F}_3(t) = \mathcal{F}_2(t) \cup \left\lbrace \text{SocialProximityToCases}(t) \right\rbrace
        \end{equation*}
    \end{block}
\end{frame}

\subsection{Datasets}

\begin{frame}{Covid-19 cases time series}
    \begin{table}[h]
    \centering
    \begin{tabular}{| c | c | c | c | c |}
        date & infective & confirmed & recoveries & deaths \\
        \hline\hline
        1/22/20 & 0 & 0 & 0 & 0 \\
        \hline
        1/23/20 & 2 & 2 & 0 & 0 \\
        \hline
        1/24/20 & 2 & 2 & 0 & 0 \\
        \hline
        $\cdots$ & $\cdots$ & $\cdots$ & $\cdots$ & $\cdots$ \\
    \end{tabular}
    \caption{Structure of the processed Covid-19 time series datasets}.
    \label{tab:country-covid-timeseries}
    \end{table}

    \begin{itemize}
        \item \gls{JHU} Covid-19 public datasets
        \item \textit{VNExpress} Covid-19 public dashboard
        \item \textit{VnCDC} Covid-19 public dashboard
    \end{itemize}
\end{frame}

\begin{frame}{Facebook's movement range maps dataset}
    \begin{table}[h]
    \centering
    \begin{tabular}{| c | c | c | c | c |}
        ds & country & polygon & rel. change & stay put ratio \\
        \hline\hline
        $\cdots$ & $\cdots$ & $\cdots$ & $\cdots$ & $\cdots$ \\
        \hline
        2021-01-01 & VNM & VNM.1.10\_1 & 0.125 & 0.270 \\
        \hline
        2021-01-02 & VNM & VNM.1.10\_1 & 0.052 & 0.259 \\
        \hline
        2021-01-03 & VNM & VNM.1.10\_1 & 0.185 & 0.269 \\
        \hline
        $\cdots$ & $\cdots$ & $\cdots$ & $\cdots$ & $\cdots$ \\
    \end{tabular}
    \caption{Structure of Facebook's \glspl{MRM} dataset.}
    \label{tab:facebook-movement-range-maps}
    \end{table}
\end{frame}

\begin{frame}{Facebook's social connectedness index}
    \begin{block}<1->{\glsfirst{SCI}}
        \begin{equation*}
            \text{SCI}_{i,j} = \frac{\text{FB connections}_{i,j}}{\text{FB users}_i * \text{FB users}_j},
        \end{equation*}
    \end{block}

    \begin{block}<2->{\glsfirst{SPC} index \cite{kuchlerGeographicSpreadCOVID192020}}
        \begin{equation*}
            \text{SPC}_{i,t} = \sum_j^C \text{Cases per 10k}_{j,t} \frac{\text{SCI}_{i,j}}{\sum_h^C \text{SCI}_{i,h}},
        \end{equation*}
    \end{block}
\end{frame}

\begin{frame}{Population data}
    \begin{table}[h]
    \centering
    \begin{tabular}{| c | c | c |}
        ID\_1 & NAME\_1 & AVGPOPULATION \\
        \hline\hline
        3 & Ha Noi & 8.2466e6 \\
        \hline
        62 & Vinh Phuc & 1.1712e6 \\
        \hline
        $\cdots$ & $\cdots$ & $\cdots$ \\
        \hline
        1001.0 & Autauga, Alabama, US & 55869 \\
        \hline
        1003.0 & Baldwin, Alabama, US & 223234 \\
        \hline
        $\cdots$ & $\cdots$ & $\cdots$ \\
    \end{tabular}
    \caption{Structure of the processed average population datasets}.
    \label{tab:country-subdivision-population}
    \end{table}

    \begin{itemize}
        \item \gls{GSO}
        \item \gls{JHU} Covid-19 datasets
    \end{itemize}
\end{frame}

\subsection{Parameters estimation}

\begin{frame}{Loss function}
    \begin{block}{Regularized \gls{MSE} with scaled outputs}
        \begin{equation*}
            \begin{aligned}
            \mathcal{L}(\hat{y}, y)
            &= \frac{1}{T} \sum_{i=1}^N \sum_{t=0}^{T-1} \left[ e^{\zeta t} \left(\frac{\hat{y}_{i,t} - y_{i,t}}{max(y_i) - min(y_i)}\right)^2\right]\\
            &+ \frac{\lambda}{2T} (\Vert\theta_1\Vert^2_2 + \Vert\theta_2\Vert^2_2)
            \end{aligned}
            \label{eq:ude-model-loss}
        \end{equation*}
    \end{block}
\end{frame}

\begin{frame}{Training process}
    \begin{figure}[h]
        \centering
        \includegraphics[scale=0.25]{ude-covid19vn-training-flow.png}
        \caption[Training procedure]{Model training procedure}
        \label{fig:model-training-flow}
    \end{figure}
\end{frame}

% \subsection{Evaluation metrics}

% \begin{frame}{Error functions}
%     \begin{block}{Mean absolute error}
%         \begin{equation*}
%             MAE = \frac{1}{n} \sum_{i=1}^n \left| \hat{y}_i - y_i \right|
%         \end{equation*}
%     \end{block}

%     \begin{block}{Mean absolute percentage error}
%         \begin{equation*}
%             MAPE = \frac{100}{n} \sum_{i=1}^n \left| \frac{\hat{y}_i - y_i}{y_i} \right|
%         \end{equation*}
%     \end{block}

%     \begin{block}{Root mean squared error}
%         \begin{equation*}
%             RMSE = \sqrt{\frac{1}{n} \sum_{i=1}{n} (\hat{y}_i - y_i)^2}
%         \end{equation*}
%     \end{block}
% \end{frame}

\subsection{Experiments}

\begin{frame}{Data preprocessing}
    \begin{itemize}
        \item<1-> Applied 7-day moving average to all datasets
        \item<1-> Applied min-max scaling on the \glsfirst{MRM} dataset and \glsfirst{SPC} index
        \item<2-> Training period: 48 days
        \begin{itemize}
            \item Vietnam: First date when the total confirmed cases passed 500
            \item United States: 1st July 2021
        \end{itemize}
        \item<3-> Testing period: 28 days
    \end{itemize}
\end{frame}

\begin{frame}{Differential equation solver}
    \begin{itemize}
        \item Tsit5 solver (Tsitouras Runge-Kutta 5/4 method \cite{tsitourasRungeKuttaPairs2011})
        \item InterpolatingAdjoint technique for approximating gradients in \gls{NeuralODE} \cite{daulbaevInterpolationTechniqueSpeed2020}
    \end{itemize}
\end{frame}

\begin{frame}{Initial conditions}
    \begin{table}[h]
        \centering
        \begin{tabular}{| c | c | c | c | c |}
            Location & $S(0)$ & $E(0)$ & $I(0)$ & $R(0)$ \\
            \hline\hline
            Vietnam & 9.75e7 & 25 & 5 & 2817 \\
            \hline
            Ho Chi Minh city & 9.22e6 & 173.5 & 34.7 & 360.1 \\
            \hline
            Binh Duong & 2.58e6 & 206.4 & 41.2 & 314.7 \\
            \hline
            Dong Nai & 3.17e6 & 295 & 59 & 194.8 \\
            \hline
            Long An & 1.71e6 & 217.8 & 43.5 & 300.5 \\
            \hline
            United States & 2.99e8 & 71890 & 14378 & 3.31e7 \\
            \hline
            Los Angeles, California & 8.78e6 & 2500 & 500 & 1.22e6 \\
            \hline
            Cook, Illinois & 4.59e6 & 615 & 123 & 546508 \\
            \hline
            Harris, Texas & 4.30e6 & 930 & 186 & 396430 \\
            \hline
            Maricopa, Arizona & 3.92e6 & 1325 & 265 & 549899 \\
            \hline
        \end{tabular}
        \caption{Initial conditions for the system of \glspl{ODE}}
        \label{tab:ude-model-initial-conditions}
    \end{table}
\end{frame}

\begin{frame}{Initial parameters}
    \begin{table}[h]
        \centering
        \begin{tabular}{| c | c | c | c |}
            Parameter & Value & Lower bound & Upper bound \\
            \hline\hline
            $\beta$ & \textit{N/A} & $0.05$ & $1.67$ \\
            \hline
            $\gamma$ & $1/4$ & $1/4$ & $1/4$ \\
            \hline
            $\lambda$ & $1/14$ & $1/14$ & $1/14$ \\
            \hline
            $\alpha$ & \textit{N/A} & $0.005$ & $0.05$ \\
            \hline
            $\theta_1$ & \textit{glorot\_normal} & \textit{N/A} & \textit{N/A} \\
            \hline
            $\theta_2$ & \textit{glorot\_normal} & \textit{N/A} & \textit{N/A} \\
            \hline
        \end{tabular}
        \caption{Initial parameters for the system of \glspl{ODE}}
        \label{tab:ude-model-initial-parameters}
    \end{table}
\end{frame}
